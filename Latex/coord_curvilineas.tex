
\chapter{Constantes fundamentais da F�sica}

\section{Constantes Fundamentais}
\begin{table}[h!]
\centering
\begin{tabular}{ccc}
\hline \hline
S�mbolo & Valor & Descri��o   \\
\hline
\hline $c$ & $3,00 \times 10^8$ m/s & velocidade da luz no v�cuo  \\ 
\hline $q$ & $1,60 \times 10^{-19}$ C & carga fundamental do el�tron    \\
\hline $m_0$ & $9,11 \times 10^{-31}$ kg & massa do el�tron em repouso   \\  
\hline $\hbar$ & $6,58 \times 10^{-16}$ eVs & constante de Planck   \\  
\hline $k_B$ & $1,38 \times 10^{-23}$ J/K & constante de Boltzmann   \\  
\hline $\epsilon_0$ & $8,85 \times 10^{-12}$ C$^2$/(Nm$^{2}$) & permissividade el�trica do v�cuo \\
\hline $\mu_0$ & $4\pi \times 10^{-7}$ N/A$^2$ & permeabilidade magn�tica do v�cuo \\
\hline $G$ & $6.674 \times 10^{-11}$ Nm$^2$/kg$^2$ & constante gravitacional universal\footnote{Medido por Henry Cavendish em 1798.} \\
\hline
\hline
\end{tabular}
\caption[Constantes fundamentais da F�sica.]{Constantes fundamentais da F�sica utilizadas neste trabalho.}
\label{constantes-fundamentais}
\end{table}
